Ferromagnetic (FM) materials became an irreducible part of modern technology. Invariable usage of FM in power generators, engines as well as mass-market consumer products like cellphones, laptops, microphones, and hard drives have led to an increase in the demand for these materials hundreds of times over the past few decades. A wast majority of used now magnets contains rare earth elements (REEs) in its composition. Such structures proved to show extremely high magnetic properties at room temperature within a relatively low price per energy-density unit. But pricing for REEs starts to grow in the last few years rapidly.
The reasoning for it is a strong monopoly of China in which, according to the World Trade Organization data, located more than 95\% of world REEs mining. This situation is quite unsatisfying for many counties with high demand for these materials. And also raises a great interest of science and industry in discovering new REEs-free FM structures. 

Nevertheless, despite several attempts to computationally predict new structures made in the last 25 years, this problem remains unsolved. One reason for it is a limitation of modern algorithms, which allows to exact only elementary properties like stability and magnetic moment, staying absolutely blind to FM critical temperature. Consequently, a huge amount of computational and experimental efforts results in materials with low technological potential due to unsatisfying values of Curie temperature. This problem indicates the need for the development of an effective and automated way for critical temperature estimation, which will empower the computational search of magnetic structures.

To sum up, in this work will be performed a computational search for the novel composition of ferromagnetic materials not containing REEs and estimation of their critical temperature by the newly developed method in order to find the most promising materials from the technological viewpoint.