Modern technology would be impossible without hard magnetic materials which play an important role in the advancement of industrial and scientific growth. They are invariably used in power generation and transmission, analogue and digital data storage, medical appliances like magnetic resonance imaging (MRI), magnetic therapy and drug delivery, sensors and actuators, scientific instruments, etc. 

Despite this in the past 25 years, there has been no significant achievement in the discovery of new hard magnets and all the existing ones contain rare-earth elements (REEs) in their composition. Also interesting fact that due to the World Trade Organisation data, more than 90\% of world rare-earth-elements mining situated in China. This situation makes rare-earth elements critical materials, and also, it's mean that all magnets production in the world strongly dependent on China suppliers. This situation is not satisfying for many countries which economy are strongly relay on these materials and also lead to a challenging task for specialists in material science to find a possible composition of strong hard magnets that will not contain REEs.


In the last few years, there have been a few attempts at systematically predicting the existence of new magnets applying various computational techniques. Unfortunately, all existing modern methods can exact only some elementary properties (stability, magnetic moment per cell, density of states, magnetic-crystalline anisotropy, etc.), staying absolutely blind to the critical temperature. As a consequence, most of the discovery computational efforts are usually spent on the materials with unsatisfying $T_{C}$ and therefore with low technological potential. Hence Curie temperature is one of the most important optimization parameters in a computational search for novel magnetic materials. 

To sum up in this work will be performed a computational search for the novel composition of ferromagnetic materials not containing REEs and estimation of their critical temperature by the newly developed method in order to find the most promising materials from the technological viewpoint.