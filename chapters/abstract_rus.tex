\selectlanguage{russian}
В наши дни сложно переоценить роль магнитных материалов в индустриальном и научном развитии. Повсеместное использование ферромагнетиков в тяжелой промышленности, цифровых приборах, накопителях (жесткие диски) и медицинском оборудовании (МРТ) делают этот класс материалов критически важным.

Несмотря на это, в последние 25 лет не было сделано никаких значительных открытий в области поиска новых ферромагнетиков. В то же время, наиболее широко используемые на сегодняшний день составы магнитных материалов неизменно включают в себя редкоземельные элементы (РЗЭ).  Принимая во внимание, чрезвычайно неравномерное распределениие добычи РЗЭ,  на более чем 90\% сконцентрированной в Китае, cложившаяся ситуация является не удовлетворительной для многих стран, экономика которых напрямую зависит от импорта этих материалов.
Совокупность вышеназванных фактов ясно обозначает задачу поиска эффективной альтернативы уже существующим магнитам, состав которых не включал бы в себя критические или дорогие элементы.

Безусловно, в последние несколько лет было предпринято несколько попыток систематического поиска новых магнитов с использованием современных вычислительных алгоритмов, но, к сожалению, все разработанные на данный момент методы могут определять только некоторые элементарные свойства материала такие как: стабильность, магнитный момент на ячейку и т. д., при этом не давая данных о критической температуре.  Как следствие, большая часть вычислительных ресурсов тратится не рационально: на материалы с неудовлетворительно низкой критической температурой и как следствие малым или отсутсвующим технологическим потенциалом. Основываясь на вышесказанном, можно заключить, что на сегодняшний день температура Кюри является чрезвычайно важным параметром при оптимизации в вычислительном поиске новых магнитных материалов и объясняет актуальность разработки эффективного метода ее определения.

В представленной работе будет проведен вычислительный поиск новых структур ферромагнитных материалов, отвечающих обозначенным физическим и экономическим критериям с использованием эволюционного алгоритма. Для обнаруженных структур будет проведен расчет их критической температуры и магнитной анизотропии с целью их ранжирования и определения наиболее перспективных с технологической точки зрения. 

